\section*{Grant Proposals}

\subsection*{Externally Funded}% -- \textbf{PI Total: \$2.4M,  co-PI Total: \$10.3M}} 

\ifdefined\iscockrell
{\footnotesize
  \begin{center}
    \begin{longtable}{p{1.7cm}p{6.5cm}p{3cm}P{1.5cm}p{1.0cm}}
      \centering\textbf{Role and} \newline \textbf{Co-Investigators}  & \centering\textbf{Title} & \centering\textbf{Agency} & \centering\textbf{Grant Total} \newline \textbf{(My Share)} & \textbf{Grant Period} \\
      \midrule
      PI & Sandia X-Prize Necking Challenge & Sandia National Laboratories & \$44,700 & 1/2012--12/2012  \\
      PI & Peridynamic Simulation of Granular Materials Undergoing Shock Compression & Sandia National Laboratories & \$32,597 & 1/2012--12/2012 \\
      PI & Statistical coarse--graining of molecular dynamics into peridynamics & \textit{Subaward} from ARL via Johns Hopkins University & \$91,925 & 1/2012--12/2012 \\
      co-PI \newline Sharma, \newline M. (PI) & Fracture Design, Placement And Sequencing In Horizontal Wells, DE-FOA-0000724 & National Energy Technology Laboratory & \$1,592,451 \newline (\$275,250) & 1/2012--12/2016 \\% &  &  } &  \\
      PI & Peridynamic simulation of pressure-shear experiments on granular media &  Sandia National Laboratories & \$29,071 & 1/2013--12/2013 \\
      PI & Towards a multiscale failure modeling paradigm for polymers: statistical coarse-graining of molecular dynamics into peridynamics. & \textit{Subaward} from ARL via The Johns Hopkins University & \$91,925 & 1/2013--12/2013 \\
      PI & Predictive simulation of material failure using peridynamics-advanced constitutive modeling, verification, and validation, BAA-AFOSR-2012-0001  & AFOSR & \$360,000. & 9/2013-12/2015 \\
      PI & Fiber failure modeling with peridynamics & \textit{Subaward} from ARL via The Johns Hopkins University & \$101,306. & 1/2013--12/2013 \\
      co-PI \newline Sharma, \newline M. (PI) & GRA support for hydraulic fracture modeling with peridynmaics (Jason York) & IAP on Hydraulic Fracturing & \$39,819 & 1/2015--12/2015 \\
      \rowcolor{Gray!40}
      co-PI \newline Madenci, \newline E. (PI) \newline Bobaru, \newline F. (co-PI) \newline Chawla, \newline N. (co-PI) \newline Du, \newline Q. (co-PI) & MURI Center for Material Failure Prediction Through Peridynamics, ONRBAA12-020  & AFOSR & \$7,500,000 \newline  (\$959,153) & 9/2013--12/2018 \\
      \rowcolor{Gray!50}
      PI & Nonlocal and fractional order methods for near--wall turbulence, large-eddy simulation, and fluid--structure interaction, ONRFOA14-012 & ARL & {\$}345,000. &  1/2015--12/2018 \\
      \rowcolor{Gray}
      PI & Pulse Fracture Simulation & GE Global Research & \$100,000. & 1/2016--12/2006 \\
      \rowcolor{Gray}
      co-PI \newline Sharma, \newline M. (PI) & GRA support for hydraulic fracture modeling with peridynmaics (Jason York) & IAP on Hydraulic Fracturing & \$33,845 & 1/2015--12/2015 \\
      \rowcolor{Gray}
      co-PI \newline Sharma, \newline M. (PI) & GRA support for hydraulic fracture modeling with peridynmaics (Shivam Agrawal) & IAP on Hydraulic Fracturing  & \$67,664 \newline (\$33,832) & 1/2016--12/2019 \\
      \rowcolor{Gray}
      PI & Student and Postdoc Travel Support for the 15th USNCCM in Austin, TX &  NSF, Proj. No. 1935320. & \$25,000 & 1/2019--8/2019 \\
      \rowcolor{Gray}
      PI &  ASCeND: ASymptotically Compatible strong form foundations for Nonlocal Discretization. &  Sandia National Laboratories & \$300,000 & 9/2018-10/2021 \\
      \rowcolor{Gray}
      PI & MATNIP: Mathematical Foundations of Nonlocal Interface Problems & Sandia National Laboratories &  \$121,000 & 9/2020-10/2022 \\
      \rowcolor{Gray}
      PI \newline Pyrcz, \newline M. (co-PI) & DiReCT: Digital Reservoir Characterization Technology & DiReCT IAP & \$900,000 \newline (\$300,000) & 7/2019--present \\
      \rowcolor{Gray}
      PI & Assessing Capillary End Effects on Large Scale Tight Reservoir Drainage &  American Chemical Society &  \$110,000 & 1/2021--1/2023 \\
      \midrule
      {} & {} & \textbf{Totals:} & \$11,541,303 \newline (\$3,049,423) & {} \\
        \cellcolor{Gray} & Indicates awarded in rank & {} &\cellcolor{Gray} \$1,236,509 \newline (\$602,677) & \\
        \cellcolor{Gray!40} & Indicates significant reseach spending in rank &  {} &\cellcolor{Gray!40} \$5,966,509 \newline (\$1,408,169)  & \\
      \bottomrule
    \end{longtable}
  \end{center}
}
\else
\ifdefined\ispdf
\begin{etaremune}
    \item Assessing Capillary End Effects on Large Scale Tight Reservoir Drainage.  American Chemical Society, 2021-2023. \textit{PI} \$110,000.
    \item MATNIP: Mathematical Foundations of Nonlocal Interface Problems. Sandia National Laboratories, 2020-2022. \textit{PI} \$121,000. 
    \item DiReCT: Digital Reservoir Characterization Technology IAP, 2020-23. \textit{co-PI w/ M. Pyrcz, E. van Oort, C. Torres-Verdin} Total Award to-date (15 member payments): \$900,000 Foster Award to-date: \$300,000.
    \item ASCeND: ASymptotically Compatible strong form foundations for Nonlocal Discretization. Sandia National Laboratories, 2018-2021. \textit{PI} \$300,000.
    \item ``Student and Postdoc Travel Support for the 15th USNCCM in Austin, TX.''  National Science Foundation, 2019, Proj. No. 1935320. PI \$25,000
  \item IAP on Hydraulic Fracturing.  Support for GRA 2016. \textit{co-PI w/ M. Sharma (UT-Austin)} \$33,845
  \item IAP on Hydraulic Fracturing.  Support for GRA (Shivam Agrawal) 2016-2019. \textit{co-PI/co-advisor w/ M. Sharma (UT-Austin)} \$33,832 (1/2 of total support)
    \item Pulse Fracture Simulation. GE Global Research, 2016. \textit{PI} \$100,000.
\else
\begin{enumerate}
\fi
    \item Assessing Capillary End Effects on Large Scale Tight Reservoir Drainage.  American Chemical Society, 2021-2023. \textit{PI} \$110,000.
  \item MATNIP: Mathematical Foundations of Nonlocal Interface Problems. Sandia National Laboratories, 2020-2022. \textit{PI} \$121,000. 
    \item DiReCT: Digital Reservoir Characterization Technology IAP, 2020-23. \textit{co-PI w/ M. Pyrcz, E. van Oort, C. Torres-Verdin} Total Award to-date: \$900,000 Foster Award to-date: \$300,000.
  \item ASCeND: ASymptotically Compatible strong form foundations for Nonlocal Discretization. Sandia National Laboratories, 2018-2021. \textit{PI} \$300,000.
    \item ``Student and Postdoc Travel Support for the 15th USNCCM in Austin, TX.''  National Science Foundation, 2019, Proj. No. 1935320. PI \$25,000, 2019
  \item IAP on Hydraulic Fracturing.  Support for GRA (Shivam Agrawal) 2016-2019. \textit{co-PI/co-advisor w/ M. Sharma (UT-Austin)} \$33,832 (1/2 of total support)
    \item Pulse Fracture Simulation. GE Global Research, 2016. \textit{PI} \$100,000.
  \item IAP on Hydraulic Fracturing.  Support for GRA (Jason York) 2016. \textit{co-PI w/ M. Sharma (UT-Austin)} \$33,845
  \item IAP on Hydraulic Fracturing.  Support for GRA (Jason York) 2015. \textit{co-PI w/ M. Sharma (UT-Austin)} \$39,819
  \item Nonlocal and fractional order methods for near-wall turbulence, large-eddy simulation, and fluid--structure interaction. Army Research Office, 2015-2018. ONRFOA14-012, \textit{PI} {\$}345,000.
  \item Fiber failure modeling with peridynamics. \textit{Subaward} from Army Reasearch Laboratories Materials in Extreme Dynamic Environments Cooperative Research Agreement.  The Johns Hopkins University, 2014. \textit{PI} \$101,306.
  \item MURI Center for Material Failure Prediction Through Peridynamics. Air Force Office of Scientific Research, 2013-2018. ONRBAA12-020, \textit{co-PI w/ E. Madenci (Arizona), F. Bobaru (Nebraska), N. Chawla (Arizona State), Q.\ Du (Columbia)} Total Award {\$}7,500,000.  Foster Award: \$959,153.
  \item Predictive simulation of material failure using peridynamics-advanced constitutive modeling, verification, and validation. Air Force FY2013 Young Investigator Program. BAA-AFOSR-2012-0001, AFOSR, 2013-2015. \textit{PI} \$360,000.
  \item Towards a multiscale failure modeling paradigm for polymers: statistical coarse-graining of molecular dynamics into peridynamics. \textit{Subaward} from Army Reasearch Laboratories Materials in Extreme Dynamic Environments Cooperative Research Agreement.  The Johns Hopkins University, 2013. \textit{PI} \$91,925.
  \item Peridynamic simulation of pressure-shear experiments on granular media.  Sandia National Laboratories, 2013. \textit{PI} \$29,071
  \item Fracture Design, Placement And Sequencing In Horizontal Wells. National Energy Technology Laboratory 2012-2016, DE-FOA-0000724 \textit{co-PI w/ M. Sharma (UT-Austin)} Total Award: {\$1,592,451}, Foster Award: \$275,250.
  \item Statistical coarse-graining of molecular dynamics into peridynamics. \textit{Subaward} from Army Reasearch Laboratories Materials in Extreme Dynamic Environments Cooperative Research Agreement.  The Johns Hopkins University, 2012. \textit{PI} \$91,925.
  \item Peridynamic Simulation of Granular Materials Undergoing Shock Compression.  Sandia National Laboratories, 2012. \textit{PI} \$32,597
  \item Sandia X-Prize Necking Challenge.  Sandia National Laboratories, 2012. \textit{PI} \$44,700.
\ifdefined\ispdf
\end{etaremune}
\else
\end{enumerate}
\fi
\fi


\subsection*{Internally Funded}

\begin{enumerate}
    \item Moncrief Grand Challenge: GFEM Framework for Reservoir Simulation of Unconventionals. Institute for Computational Engineering and Sciences, 2018. \textit{PI} \$75,000
  \item Application of Peridynamics to Hydraulic Fracture Modeling. The University of Texas at San Antonio -- Office of the Vice President for Research, 2012. \textit{PI} \$18,927.
\end{enumerate}


\subsection*{Pending}

\begin{enumerate}
  \item SciML at CAMINO: Sandia Supplementary University Partnerships Proposal. Sandia National Laboratories. 2023-2026. \textit{PI}. Requested \$525,000.
\item DOE Earthshot: Center for Multiscale Mechanics \& Flow.  Department of Energy LAB 23-2954, 2023-2027. Co-PI. Requested \$20MM, Foster share: \$909,274
\item Calcined Petroleum Coke as a High-Temperature Diagnostic Proppant for Geothermal Applications.  Department of Energy DE-EE0007080, 2023-2026. Co-PI. Requested \$1.5M
\item Hydrogen storage in salt caverns in the Permian Basin:  Seal integrity evaluation and field test.  Department of Energy DE-FOA-0002400, 2023-2025. Co-PI. Requested \$1.5M
\end{enumerate}
    %\item CAREER: A nonlocal approach to fluid driven fracture with applications in energy production and environmental assessment. National Science Foundation, 2016-2020. Requested \$500,000.
    %\item Robust discretizations for nonlocal mechanics. Office of Naval Research ONR-15-FOA-0006, 2015-2018. \textit{PI} Total Award {\$}510,000.
  %\item Collaborative Research: A projection method for constraint-free plasticity model integration. National Science Foundation, 2015-2018. \textit{PI} Total Requested: \$451,054, Foster Requesting: \$190,519.
    %\item Bridging the length scales through a unified nonlocal multiscale framework. National Science Foundation, 2014-2017. \textit{PI} requested \$234,407.
    %\item DOE Career: Nonlocal porous flow in evolving fractured media using peridynamic theory. Department of Energy, 2014-2017. \textit{PI} Requested: \$750,000. 
  %\item DTRA Young Investigator Program: Multiscale peridynamic simulation of geomaterials under impact loading. Defense Threat Reduction Agency, 2014-2016. \textit{PI} Reqesting: \$200,000. (Rated meritorious, but funding withheld subject to availability.)
    %\item DOE Career: Nonlocal porous flow in evolving fractured media using peridynamic theory. Department of Energy, 2013-2016. Requesting: \$749,875. 
    %\item  Investigating Cellular And Subcellular Behaviors and Metabolic Mechanisms Using Thermal and Raman Imaging Techniques, National Science Foundation, 2013-2016.  Co-PI Requesting: \$705,052
    %\item BRIGE: A nonlocal mixture theory approach to fluid driven fracture with applications in energy production and environmental assessment. National Science Foundation, 2013-2015. Requested \$174,702.
    %\item Dynamic Failure Mechanisms of Advanced Fiber Materials. Joint proposal with SwRI to the SwRI/UTSA CONNECT program, 2013. Requested \$99,940.
    %\item Towards exascale computational mechanics: exploiting the newest generation of heterogenous HPC clusters. Oak Ridge Associated Universities Ralph E. Powe Junior Faculty Enhancement Award. Oak Ridge National Laboratories, 2013. Requesting: \$10,000. 
    %\item BRIGE: Identification and Simulation of Non-Local Effects to Improve Predictive Analysis of Heterogenous Materials. National Science Foundation, 2012-2014. Requested \$174,805.
    %\item Discontinuous Flow and Angled Localization: Modern Challenges in Material Failure. Joint proposal with SwRI to the SwRI/UTSA CONNECT program, 2012. Requested \$93,780.
    %\item A novel torsional Kolsky bar for testing materials at constant shear strain rates. Haythornthwaite Research Initiation Grant Program, 2011. Requested \$13,388.
    %\item Joint proposal with SwRI in response to BAA AFOSR 2011-06 on University Center of Excellence: High-rate Deformation Physics of Heterogeneous Materials. AFOSR, 2011. Total Proposed: \$5,000,000, Foster Requested: \$377,518.
%\end{enumerate}
