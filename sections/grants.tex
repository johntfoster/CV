\section*{Grant Proposals}

\subsection*{Externally Funded}% -- \textbf{PI Total: \$2.4M,  co-PI Total: \$10.3M}} 

\ifdefined\iscockrell
    \pagebreak[2]
    \textbf{In rank of Assistant Professor}
\begin{enumerate}
  \item Sandia X-Prize Necking Challenge.  Sandia National Laboratories, 2012. \textit{PI} \$44,700.
  \item Peridynamic Simulation of Granular Materials Undergoing Shock Compression.  Sandia National Laboratories, 2012. \textit{PI} \$32,597
  \item Statistical coarse-graining of molecular dynamics into peridynamics. \textit{Subaward} from Army Reasearch Laboratories Materials in Extreme Dynamic Environments Cooperative Research Agreement.  The Johns Hopkins University, 2012. \textit{PI} \$91,925.
  \item Fracture Design, Placement And Sequencing In Horizontal Wells. National Energy Technology Laboratory 2012-2016, DE-FOA-0000724 \textit{co-PI w/ M. Sharma (UT-Austin)} Total Award: {\$1,592,451}, Foster Award: \$275,250.
  \item Peridynamic simulation of pressure-shear experiments on granular media.  Sandia National Laboratories, 2013. \textit{PI} \$29,071
  \item Towards a multiscale failure modeling paradigm for polymers: statistical coarse-graining of molecular dynamics into peridynamics. \textit{Subaward} from Army Reasearch Laboratories Materials in Extreme Dynamic Environments Cooperative Research Agreement.  The Johns Hopkins University, 2013. \textit{PI} \$91,925.
  \item Predictive simulation of material failure using peridynamics-advanced constitutive modeling, verification, and validation. Air Force FY2013 Young Investigator Program. BAA-AFOSR-2012-0001, AFOSR, 2013-2015. \textit{PI} \$360,000.
  \item MURI Center for Material Failure Prediction Through Peridynamics. Air Force Office of Scientific Research, 2013-2018. ONRBAA12-020, \textit{co-PI w/ E. Madenci (Arizona), F. Bobaru (Nebraska), N. Chawla (Arizona State), Q.\ Du (Columbia)} Total Award {\$}7,500,000.  Foster Award: \$959,153.
  \item Fiber failure modeling with peridynamics. \textit{Subaward} from Army Reasearch Laboratories Materials in Extreme Dynamic Environments Cooperative Research Agreement.  The Johns Hopkins University, 2014. \textit{PI} \$101,306.
  \item Nonlocal and fractional order methods for near-wall turbulence, large-eddy simulation, and fluid--structure interaction. Army Research Office, 2015-2018. ONRFOA14-012, \textit{PI} {\$}345,000.
  \item Pulse Fracture Simulation. GE Global Research, 2016. \textit{PI} \$100,000.
\end{enumerate}
    \pagebreak[2]
    \textbf{In rank of Associate Professor at UT-Austin}
\begin{enumerate}[resume]
    \item ``Student and Postdoc Travel Support for the 15th USNCCM in Austin, TX.''  National Science Foundation, Proj. No. 1935320. PI \$25,000
    \item ASCeND: ASymptotically Compatible strong form foundations for Nonlocal Discretization. Sandia National Laboratories, 2018-2021. \textit{PI} \$300,000.
    \item DiReCT: Digital Reservoir Characterization Technology IAP, 2020-22. \textit{co-PI w/ M. Pyrcz, E. van Oort, C. Torres-Verdin} Total Award to-date: \$600,000 Foster Award to-date: \$200,000.
    \item MATNIP: Mathematical Foundations of Nonlocal Interface Problems. Sandia National Laboratories, 2020-2022. \textit{PI} \$121,000. 
    \item Assessing Capillary End Effects on Large Scale Tight Reservoir Drainage.  American Chemical Society, 2021. \textit{PI} \$110,000.
\end{enumerate}
\else
\ifdefined\ispdf
\begin{etaremune}
    \item Assessing Capillary End Effects on Large Scale Tight Reservoir Drainage.  American Chemical Society, 2021. \textit{PI} \$110,000.
    \item MATNIP: Mathematical Foundations of Nonlocal Interface Problems. Sandia National Laboratories, 2020-2022. \textit{PI} \$121,000. 
    \item DiReCT: Digital Reservoir Characterization Technology IAP. 2020-21. \textit{co-PI w/ M. Pyrcz, E. van Oort, C. Torres-Verdin} \$240,000.
    \item ASCeND: ASymptotically Compatible strong form foundations for Nonlocal Discretization. Sandia National Laboratories, 2018-2021. \textit{PI} \$300,000.
    \item ``Student and Postdoc Travel Support for the 15th USNCCM in Austin, TX.''  National Science Foundation, Proj. No. 1935320. PI \$25,000
    \item Pulse Fracture Simulation. GE Global Research, 2016. \textit{PI} \$100,000.
\else
\begin{enumerate}
\fi
    \item Assessing Capillary End Effects on Large Scale Tight Reservoir Drainage.  American Chemical Society, 2021. \textit{PI} \$110,000.
  \item MATNIP: Mathematical Foundations of Nonlocal Interface Problems. Sandia National Laboratories, 2020. \textit{PI} \$121,000. 
    \item DiReCT: Digital Reservoir Characterization Technology IAP, 2020-22. \textit{co-PI w/ M. Pyrcz, E. van Oort, C. Torres-Verdin} Total Award to-date: \$600,000 Foster Award to-date: \$200,000.
  \item ASCeND: ASymptotically Compatible strong form foundations for Nonlocal Discretization. Sandia National Laboratories, 2018-2021. \textit{PI} \$300,000.
    \item ``Student and Postdoc Travel Support for the 15th USNCCM in Austin, TX.''  National Science Foundation, Proj. No. 1935320. PI \$25,000
  \item Nonlocal and fractional order methods for near-wall turbulence, large-eddy simulation, and fluid--structure interaction. Army Research Office, 2015-2018. ONRFOA14-012, \textit{PI} {\$}345,000.
  \item Fiber failure modeling with peridynamics. \textit{Subaward} from Army Reasearch Laboratories Materials in Extreme Dynamic Environments Cooperative Research Agreement.  The Johns Hopkins University, 2014. \textit{PI} \$101,306.
  \item MURI Center for Material Failure Prediction Through Peridynamics. Air Force Office of Scientific Research, 2013-2018. ONRBAA12-020, \textit{co-PI w/ E. Madenci (Arizona), F. Bobaru (Nebraska), N. Chawla (Arizona State), Q.\ Du (Columbia)} Total Award {\$}7,500,000.  Foster Award: \$959,153.
  \item Predictive simulation of material failure using peridynamics-advanced constitutive modeling, verification, and validation. Air Force FY2013 Young Investigator Program. BAA-AFOSR-2012-0001, AFOSR, 2013-2015. \textit{PI} \$360,000.
  \item Towards a multiscale failure modeling paradigm for polymers: statistical coarse-graining of molecular dynamics into peridynamics. \textit{Subaward} from Army Reasearch Laboratories Materials in Extreme Dynamic Environments Cooperative Research Agreement.  The Johns Hopkins University, 2013. \textit{PI} \$91,925.
  \item Peridynamic simulation of pressure-shear experiments on granular media.  Sandia National Laboratories, 2013. \textit{PI} \$29,071
  \item Fracture Design, Placement And Sequencing In Horizontal Wells. National Energy Technology Laboratory 2012-2016, DE-FOA-0000724 \textit{co-PI w/ M. Sharma (UT-Austin)} Total Award: {\$1,592,451}, Foster Award: \$275,250.
  \item Statistical coarse-graining of molecular dynamics into peridynamics. \textit{Subaward} from Army Reasearch Laboratories Materials in Extreme Dynamic Environments Cooperative Research Agreement.  The Johns Hopkins University, 2012. \textit{PI} \$91,925.
  \item Peridynamic Simulation of Granular Materials Undergoing Shock Compression.  Sandia National Laboratories, 2012. \textit{PI} \$32,597
  \item Sandia X-Prize Necking Challenge.  Sandia National Laboratories, 2012. \textit{PI} \$44,700.
\ifdefined\ispdf
\end{etaremune}
\else
\end{enumerate}
\fi
\fi


\subsection*{Internally Funded}

\begin{enumerate}
    \item Moncrief Grand Challenge: GFEM Framework for Reservoir Simulation of Unconventionals. Institute for Computational Engineering and Sciences, 2018. \textit{PI} \$75,000
  \item Application of Peridynamics to Hydraulic Fracture Modeling. The University of Texas at San Antonio -- Office of the Vice President for Research, 2012. \textit{PI} \$18,927.
\end{enumerate}


\subsection*{Pending}

\begin{enumerate}
\item Calcined Petroleum Coke as a High-Temperature Diagnostic Proppant for Geothermal Applications.  Department of Energy DE-EE0007080, 2023-2026. Co-PI. Requested \$1.5M
\item Hydrogen storage in salt caverns in the Permian Basin:  Seal integrity evaluation and field test.  Department of Energy DE-FOA-0002400, 2023-2025. Co-PI. Requested \$1.5M
\end{enumerate}
    %\item CAREER: A nonlocal approach to fluid driven fracture with applications in energy production and environmental assessment. National Science Foundation, 2016-2020. Requested \$500,000.
    %\item Robust discretizations for nonlocal mechanics. Office of Naval Research ONR-15-FOA-0006, 2015-2018. \textit{PI} Total Award {\$}510,000.
  %\item Collaborative Research: A projection method for constraint-free plasticity model integration. National Science Foundation, 2015-2018. \textit{PI} Total Requested: \$451,054, Foster Requesting: \$190,519.
    %\item Bridging the length scales through a unified nonlocal multiscale framework. National Science Foundation, 2014-2017. \textit{PI} requested \$234,407.
    %\item DOE Career: Nonlocal porous flow in evolving fractured media using peridynamic theory. Department of Energy, 2014-2017. \textit{PI} Requested: \$750,000. 
  %\item DTRA Young Investigator Program: Multiscale peridynamic simulation of geomaterials under impact loading. Defense Threat Reduction Agency, 2014-2016. \textit{PI} Reqesting: \$200,000. (Rated meritorious, but funding withheld subject to availability.)
    %\item DOE Career: Nonlocal porous flow in evolving fractured media using peridynamic theory. Department of Energy, 2013-2016. Requesting: \$749,875. 
    %\item  Investigating Cellular And Subcellular Behaviors and Metabolic Mechanisms Using Thermal and Raman Imaging Techniques, National Science Foundation, 2013-2016.  Co-PI Requesting: \$705,052
    %\item BRIGE: A nonlocal mixture theory approach to fluid driven fracture with applications in energy production and environmental assessment. National Science Foundation, 2013-2015. Requested \$174,702.
    %\item Dynamic Failure Mechanisms of Advanced Fiber Materials. Joint proposal with SwRI to the SwRI/UTSA CONNECT program, 2013. Requested \$99,940.
    %\item Towards exascale computational mechanics: exploiting the newest generation of heterogenous HPC clusters. Oak Ridge Associated Universities Ralph E. Powe Junior Faculty Enhancement Award. Oak Ridge National Laboratories, 2013. Requesting: \$10,000. 
    %\item BRIGE: Identification and Simulation of Non-Local Effects to Improve Predictive Analysis of Heterogenous Materials. National Science Foundation, 2012-2014. Requested \$174,805.
    %\item Discontinuous Flow and Angled Localization: Modern Challenges in Material Failure. Joint proposal with SwRI to the SwRI/UTSA CONNECT program, 2012. Requested \$93,780.
    %\item A novel torsional Kolsky bar for testing materials at constant shear strain rates. Haythornthwaite Research Initiation Grant Program, 2011. Requested \$13,388.
    %\item Joint proposal with SwRI in response to BAA AFOSR 2011-06 on University Center of Excellence: High-rate Deformation Physics of Heterogeneous Materials. AFOSR, 2011. Total Proposed: \$5,000,000, Foster Requested: \$377,518.
%\end{enumerate}
